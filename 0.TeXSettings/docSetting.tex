%%
\documentclass[12pt, multi, tikz]{article}
%%

%% Basic Setting -----------------------------------------------------------------------
\usepackage[english]{babel}
\usepackage{amsmath}		 		% math pack
\usepackage{esint}					% Integral Symbol
\usepackage{mathtools}
\usepackage{amsthm}
\usepackage{mathrsfs}
\usepackage{empheq}

\usepackage{eufrak}
\usepackage{amssymb}


%\usepackage{dutchcal}

%\usepackage{pzccal}
%\DeclareFontFamily{OT1}{pzc}{}
%\DeclareFontShape{OT1}{pzc}{m}{it}{<-> s * [1.10] pzcmi7t}{}
%\DeclareMathAlphabet{\mathpzc}{OT1}{pzc}{m}{it}

\usepackage{cancel}
\usepackage{url}					   % add urls
\usepackage[utf8x]{inputenc}	 % latex input utf8x (various characterers)
\usepackage{natbib}				     % reference package
\usepackage{parskip}				 % space between paragraph
\usepackage{fancyhdr}				% page headers and footers
\usepackage{vmargin}				% set various marfines for header/footer and page dimension
\usepackage{fixltx2e}				  % latex debugging package
%% ----------------------------------------------------------------------- Basic Setting

%% Table Setting -----------------------------------------------------------------------
\usepackage{array}						% array pack
\usepackage{caption}  				% caption Setup 
\usepackage{chngcntr}				% caption set to follow chapter number 
\usepackage{hhline}					% Seperate line properties (tabular)
\usepackage{multirow}				% multi row package
\usepackage{booktabs,caption}	% caption
\usepackage[flushleft]{threeparttable}	 % package for tabular notes

\usepackage{makecell}

%%... Table Numbering and Format
\captionsetup[table]{labelfont=bf,  labelsep=period}	% table caption format
\captionsetup[table]{skip=5pt}		% space between caption and table
\counterwithin{table}{section}			% table caption (Chapter number)

\usepackage{graphicx}				% graph pack
\usepackage{subfig}					% sub figure
\usepackage{float}						% sub figure name

%%... Figure format
\captionsetup[figure]{labelfont=bf,  labelsep=period}	% table caption format
\counterwithin{figure}{section}	% figure caption (Chapter number)
%% ---------------------------------------------------------------------- Figure Setting

%% Equation Setting ------------------------------------------------------------------
\counterwithin{equation}{section}		% equation caption (Chapter number)

%\setlength{\abovedisplayskip}{1em}
%\setlength{\belowdisplayskip}{1em}

\usepackage[toc,page]{appendix}

%% Author and ---
\usepackage{authblk}

%% Auxillary -----------------------------------------------------------------------------
%\usepackage{subcaption}			% sub-caption 
\usepackage{bm}
%% ----------------------------------------------------------------------------- Auxillary

%% Color box for math
\usepackage[most]{tcolorbox}

\tcbset{colback=yellow!10!white, colframe=red!50!black, 
	highlight math style= {enhanced, %<-- needed for the ’remember’ options
		colframe=red,colback=red!10!white,boxsep=0pt}
}

\usepackage[draft]{todonotes}   % notes showed

%% Page Setting ------------------------------------------------------------------------
\setmarginsrb{2 cm}{1.5 cm}{2 cm}{1.5 cm}{0.5 cm}{1 cm}{0.5 cm}{1 cm}
\pagestyle{fancy}
\fancyhead[RE,LO]{\rightmark}
\fancyhead[LE,LO]{}

\newcommand{\myparagraph}[1]{\paragraph{#1}\mbox{}\\}

%1 est la marge gauche
%2 est la marge en haut
%3 est la marge droite
%4 est la marge en bas
%5 fixe la hauteur de l’entête
%6 fixe la distance entre l’entête et le texte
%7 fixe la hauteur du pied de page
%8 fixe la distance entre le texte et le pied de page

\makeatletter
\newcommand\stroke[1]{\mathpalette\stroke@aux{#1}}
\def\stroke@aux#1#2{%
	\ooalign{%
		\hfil$#1-$\hfil\cr
		\hfil$#1#2$\hfil\cr
	}%
}
\makeatother